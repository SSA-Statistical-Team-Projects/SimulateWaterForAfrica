\begin{table}
\centering
\caption{T-Test Results and Mean Differences}
\centering
\label{tab:balance_table}
\small
\begin{threeparttable}
\begin{tabular}{lccccc}
\toprule
 & \shortstack{Only \\ Rainfed} & \shortstack{Any \\ Irrigated} & Difference & p-value & Cohen's d \\
 & \multicolumn{1}{c}{(1)} & \multicolumn{1}{c}{(2)} & \multicolumn{1}{c}{(3)} & \multicolumn{1}{c}{(4)} & \multicolumn{1}{c}{(5)} \\
\midrule
Total Yield (T/Ha) & 1.627 & 3.702 & + & 0.000*** & 1.248$\dagger \dagger \dagger$\\
Growing Season: Longest Consecutive Dry Days & 83.604 & 84.775 & + & 0.000*** & 0.022\\
Length of Growing Cycle (Days) & 137.093 & 153.292 & + & 0.000*** & 0.773$\dagger \dagger$\\
Harvest Area (1000 Ha) & 169.725 & 264.957 & + & 0.000*** & 0.206$\dagger$\\
Growing Cycle: Crop Water Deficit (mm) & 16.408 & 426.009 & + & 0.000*** & 2.245$\dagger \dagger \dagger$\\
Growing Cycle: Accumulated Temp (C / Day) & 3273.826 & 3473.878 & + & 0.000*** & 0.354$\dagger$\\
Moisture Constraint in Growth & 9391.996 & 9985.325 & + & 0.000*** & 0.545$\dagger \dagger$\\
Production (1000 T) & 324.62 & 863.337 & + & 0.000*** & 0.360$\dagger$\\
PDSI Winter & -0.727 & -0.317 & + & 0.000*** & 0.089\\
PDSI Spring & -0.537 & -0.046 & + & 0.000*** & 0.109\\
PDSI Fall & -1.996 & -1.287 & + & 0.000*** & 0.183\\
PDSI Summer & -1.439 & -0.817 & + & 0.000*** & 0.154\\
Soil Suitability (Continuous Index) & 4570.515 & 6507.263 & + & 0.000*** & 0.756$\dagger \dagger$\\
Mean Depth to Groundwater (m) & 14.053 & 25.442 & + & 0.000*** & 0.377$\dagger$\\
No of Growing Season Days & 201.184 & 151.708 & - & 0.000*** & -0.608$\dagger \dagger$\\
No. of Rain Days & 149.074 & 107.439 & - & 0.000*** & -0.733$\dagger \dagger$\\
Crop Actual Evapotranspiration (mm) & 444.441 & 225.831 & - & 0.000*** & -1.640$\dagger \dagger \dagger$\\
Temp Winter & 30.851 & 30.45 & - & 0.000*** & -0.143\\
Temp Spring & 31.719 & 31.116 & - & 0.000*** & -0.175\\
Temp Fall & 28.761 & 26.417 & - & 0.000*** & -0.532$\dagger \dagger$\\
Temp Summer & 30.578 & 28.642 & - & 0.000*** & -0.341$\dagger$\\
Prec Winter & 1318.989 & 1032.755 & - & 0.000*** & -0.277$\dagger$\\
Prec Spring & 1352.468 & 979.099 & - & 0.000*** & -0.391$\dagger$\\
Prec Fall & 1112.911 & 816.357 & - & 0.000*** & -0.237$\dagger$\\
Prec Summer & 784.129 & 592.227 & - & 0.000*** & -0.182\\
Terrain Median Slope & 3.891 & 3.744 & - & 0.000*** & -0.087\\
Multi-cropping Class (With Irrigation) & 7.874 & 7.535 & - & 0.000*** & -0.504$\dagger \dagger$\\
Multi-cropping Class (With Rainfed) & 3.312 & 2.572 & - & 0.000*** & -0.507$\dagger \dagger$\\
Modified Fournier Index (mm) & 2511.226 & 2038.681 & - & 0.000*** & -0.366$\dagger$\\
\midrule
\textbf{N} & \textbf{303920} & \textbf{31346} &  &  & \\
\bottomrule
\end{tabular}
\begin{tablenotes} \footnotesize
\item Notes. In this table we compare agro-climatic factors and conditions across croplands that were solely rain-fed and croplands that practiced any level of irrigation. The analysis was conducted using data at the crop-GAEZ grid level for croplands. Columns (1) and (2) compute the averages of the outcomes for rain-fed- and irrigated- croplands, respectively. Column (3) indicates the sign difference in their means, whereas column (4) denotes whether these mean differences are statistically significant. Given that we have a large number of observations, we present Cohen's d estimates in column (5) to show the strength of the effect size. *p$<$0.1 **p$<$0.05 ***p$<$0.01. Effect size thresholds: $\dagger$ small ($d \geq 0.2$), $\dagger\dagger$ medium ($d \geq 0.5$), $\dagger\dagger\dagger$ large ($d \geq 0.8$).
\end{tablenotes}
\end{threeparttable}
\end{table}
